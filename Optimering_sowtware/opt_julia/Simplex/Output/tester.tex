\documentclass{article}
\usepackage[table]{xcolor}
\usepackage{float}
\usepackage{graphicx}
\usepackage[margin=1.5cm]{geometry}
\newsavebox{\tableauBox}
\newcommand{\fitTableau}[1]{%
\sbox{\tableauBox}{#1}%
\ifdim\wd\tableauBox>\textwidth
\resizebox{\textwidth}{!}{\usebox{\tableauBox}}%
\else
\usebox{\tableauBox}%
\fi}
\begin{document}
\begin{table}[H]
\centering
\renewcommand{\arraystretch}{1.2}
\setlength{\tabcolsep}{6pt}
\fitTableau{
\begin{tabular}{c|cccccc|c}
\textbf{Basis} & $x_1$ & $x_2$ & $x_3$ & $S_1$ & $S_2$ & $S_3$ & $b_i$ \\ \hline
$S_1$ & 1.0 & 1.0 & 1.0 & 1.0 & 0.0 & 0.0 & 10.0 \\
$S_2$ & 2.0 & 1.0 & 0.0 & 0.0 & 1.0 & 0.0 & 8.0 \\
$S_3$ & 1.0 & 2.0 & 1.0 & 0.0 & 0.0 & 1.0 & 12.0 \\
$Z$ & -5.0 & -3.0 & -2.0 & 0.0 & 0.0 & 0.0 & 0.0 \\
\end{tabular}
}

\vspace{1em}
\par\medskip
\textit{Dual simplex stoppes, fordi alle }$b_i \ge 0$ (basis er nu feasible).
\par\medskip
\par\medskip
\textit{Skifter til primal simplex, fordi basis er feasible, og vi nu optimerer objektivet.}
\par\medskip
\fitTableau{
\begin{tabular}{c|>{\columncolor{green!15}}cccccc|c|c}
\textbf{Basis} & $x_1$ & $x_2$ & $x_3$ & $S_1$ & $S_2$ & $S_3$ & $b_i$ & \textbf{Ratio} \\ \hline
$S_1$ & 0.0 & 0.5 & 1.0 & 1.0 & -0.5 & 0.0 & 6.0 & 10.0 \\
\rowcolor{green!15} $x_1$ & 1.0 & 0.5 & 0.0 & 0.0 & 0.5 & 0.0 & 4.0 & 4.0 \\
$S_3$ & 0.0 & 1.5 & 1.0 & 0.0 & -0.5 & 1.0 & 8.0 & 12.0 \\
$Z$ & 0.0 & -0.5 & -2.0 & 0.0 & 2.5 & 0.0 & 20.0 &  \\
\end{tabular}
}

\vspace{1em}
\fitTableau{
\begin{tabular}{c|cc>{\columncolor{green!15}}cccc|c|c}
\textbf{Basis} & $x_1$ & $x_2$ & $x_3$ & $S_1$ & $S_2$ & $S_3$ & $b_i$ & \textbf{Ratio} \\ \hline
\rowcolor{green!15} $x_3$ & 0.0 & 0.5 & 1.0 & 1.0 & -0.5 & 0.0 & 6.0 & 6.0 \\
$x_1$ & 1.0 & 0.5 & 0.0 & 0.0 & 0.5 & 0.0 & 4.0 &  \\
$S_3$ & 0.0 & 1.0 & 0.0 & -1.0 & 0.0 & 1.0 & 2.0 & 8.0 \\
$Z$ & 0.0 & 0.5 & 0.0 & 2.0 & 1.5 & 0.0 & 32.0 &  \\
\end{tabular}
}

\vspace{1em}
\par\medskip
\textit{Primal simplex stoppes, fordi z-rækken ikke har negative værdier (optimal løsning).}
\par\medskip
\fitTableau{
\begin{tabular}{c|cccccc|c}
\textbf{Basis} & $x_1$ & $x_2$ & $x_3$ & $S_1$ & $S_2$ & $S_3$ & $b_i$ \\ \hline
$x_3$ & 0.0 & 0.5 & 1.0 & 1.0 & -0.5 & 0.0 & 6.0 \\
$x_1$ & 1.0 & 0.5 & 0.0 & 0.0 & 0.5 & 0.0 & 4.0 \\
$S_3$ & 0.0 & 1.0 & 0.0 & -1.0 & 0.0 & 1.0 & 2.0 \\
$Z$ & 0.0 & 0.5 & 0.0 & 2.0 & 1.5 & 0.0 & 32.0 \\
\end{tabular}
}

\caption{Simplex-tableauer}
\end{table}
\end{document}
